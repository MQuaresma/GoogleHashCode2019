\documentclass{article}
\usepackage[utf8]{inputenc}
\usepackage[portuges]{babel}
\usepackage[a4paper, total={7in, 9in}]{geometry}
\usepackage{graphicx}
\usepackage{float}

\newcommand{\question}[1]{
    {\large \textbf{Q: #1}}
    \\
}

\newcommand{\titleRule}{
    \rule{\linewidth}{0.5mm} \\ [0.25cm]
}

\begin{document}

\begin{titlepage}
    \center
    \textsc{\LARGE Google Hash Code 2019 - Write Up} \\ [1.5cm]
    \titleRule
    {\huge \bfseries Solving the slideshow optimization problem with simulated annealing}
    \titleRule

    João Pedro Ferreira Vieira \\
    José Carlos Lima Martins \\
    Miguel Miranda Quaresma \\
    Simão Paulo Leal Barbosa \\[0.25cm]

    \today
\end{titlepage}

\section{Introduction}

Google Hash Code is a yearly global event organized by Google that aims to challenge teams to solve a problem by combining problem solving 
skills and programming. 
This report presents the approach taken by the \textbf{0x4a4a4d53} team, explaining the thought process behind the solution presented by the
team as well as possible alternative approaches that could have resulted in a better overall score.
The write-up will also present the computational as well as time memory complexity of the algorithms presented which might help understand
the attained performance.

\section{The Problem}
The problem presented by Google in 2019 consisted of, given a set of photos with the following attributes:
\begin{itemize}
    \item \textbf{Orientation}: Vertical (V) or Horizontal(H)
    \item \textbf{Tags}: words that described/categorized a given photo
\end{itemize}

build a slideshow (\textbf{i.e.} sequence of slides where each slide could contain one horizontal photo or two vertical ones) sorting the
slides in a way that maximized the "Interest Score" of said slideshow.

\subsection{Input file format}
The datasets to be processed consisted of text files where the first line listed the number of photos contained in the file and each subsequent
line contained a single photo's attributes separated by one (or more) spaces.
The first field indicated the orientation of the photo (\textbf{H} or \textbf{V}), followed by the number of tags that described the photos, these 
tags were listed from the third field on. Each photo also had a unique ID i where ith photos was in line i.

\subsection{Output file format}
The solution for the problem consisted of a file describing the slideshow, where the first line indicated the number of slides and the following lines
represented each individual slide containing the ID of the photo's in the respective slide.
In the cases where a slide contained two vertical photos, two photo ID's were present.


\end{document}
